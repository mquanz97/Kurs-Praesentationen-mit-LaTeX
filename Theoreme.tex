\documentclass{beamer}
\usetheme{Berlin}
\beamertemplatenavigationsymbolsempty

\begin{document}
	
	\begin{frame}
		\begin{proof}[Beweis]
			
		\end{proof}
		
		\begin{definition}
			Inhalt Definition
		\end{definition}
		
		\begin{lemma}
			Das Lemmata
		\end{lemma}
		
		\begin{theorem}
			Inhalt Theorem
		\end{theorem}
		
		\newtheorem{eigenesTheorem}{Titel von meinem eigenen Theorem}
		\begin{eigenesTheorem}
			Bemerkung
		\end{eigenesTheorem}
	\end{frame}

	\begin{frame}
	
		\begin{Satz}
			Inhalt Satz
		\end{Satz}
	
		\begin{Beweis}
			Inhalt Beweis
		\end{Beweis}
	
		\begin{Folgerung}
			Inhalt Folgerung
		\end{Folgerung}
	
		\begin{Fakt}
			Fakt
		\end{Fakt}
	\end{frame}

	\begin{frame}
		\begin{Problem}
			Inhalt Problem
		\end{Problem}
		
		\begin{Loesung}
			Lösung
		\end{Loesung}

	\end{frame}
	
	
	
\end{document}
